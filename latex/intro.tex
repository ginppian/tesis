%% Los cap'itulos inician con \chapter{T'itulo}, estos aparecen numerados y
%% se incluyen en el 'indice general.
%%
%% Recuerda que aqu'i ya puedes escribir acentos como: 'a, 'e, 'i, etc.
%% La letra n con tilde es: 'n.

\chapter{Introducci'on}

Actualmente la difusión de la tecnología ubicua sobre todo la móvil se ha extendido ampliamente gracias al uso del internet y los telefonos inteligentes o smartphones,  así la disminución de costos de los mismos. Lo anterior nos da un ecosistema fertil para un desarrollar un sin fin de aplicativos.

Cabe especificar que los smartphones cuentan con diversos sensores integrados, como lo son: acelerómetros, giroscopio, sensor de huella, sensor dactilar, bluethoot, magnetómetro, receptor GPS, etc. Y es precismente este último el que nos interesa para nuestro desarrollo.

El receptor GPS se encarga de obtener la ubicación exacta con errores mínimos pudiendonos dar: la latitud, longitud y altitud del dispositivo. Apartir de éstos datos podemos obtener información diversa como: la velocidad de desplazamiento del dispositivo, la hora exacta del dispositivo en base a su posición, la distancia a un punto dado al dispositivo, si el dispositivo ha entrado a un área en especifica, si ha salido de un área en específica. 

Con este conocimiento y la violencia e inseguridad que actualmente sufre México, buscamos que la tecnología se vuelva una aliada en el rastreo de personas, especificamente enfrentar el problema de "feminicidios" en el estado de Puebla. 

Para que las familias y seres queridos de estas personas tengan en medida de lo posible, la tranquilidad de conocer su paradero y saber que se encuentra bien y a salvo.
